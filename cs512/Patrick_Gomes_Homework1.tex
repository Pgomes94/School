\documentclass[12pt,leqno,fleqn]{article}
\usepackage{geometry}                		% See geometry.pdf to learn the layout options. There are lots.
\geometry{letterpaper}                   		% ... or a4paper or a5paper or ... 
%\geometry{landscape}                		% Activate for rotated page geometry
%\usepackage[parfill]{parskip}    		% Activate to begin paragraphs with an empty line rather than an indent
\usepackage{graphicx}				% Use pdf, png, jpg, or eps§ with pdflatex; use eps in DVI mode
								% TeX will automatically convert eps --> pdf in pdflatex		
\usepackage{amssymb}

\usepackage{prooftree}
\usepackage{boxproof}
\usepackage{parskip}

\newcommand{\Intro}[1]{{#1}{\textrm{i}}}
\newcommand{\Elim}[1]{{#1}{\textrm{e}}}

\title{CS 512, Spring 2016: Assignment 1\\Propositional Logic}
\author{Patrick Gomes}
\date{January 29, 2016}	% Activate to display a given date or no date

\begin{document}
\maketitle
\section{Problem 1}

a) $P \to Q, P \to \neg Q \vdash \neg P$
\begin{proofbox}
	\label{a1}\: P \to Q \= \textrm{premise} \\
	\label{a2}\: P \to \neg Q \= \textrm{premise} \\
	\[
		\label{a3}\: P \= \textrm{assume} \\
		\label{a4}\: Q \= \Elim{\to}\ \ \ref{a1},\ref{a3}\\
		\label{a5}\: \neg Q \= \Elim{\to}\ \ \ref{a2},\ref{a3}\\
		\label{a6}\: \bot \= \Elim{\neg}\ \ \ref{a4},\ref{a5}\\
	\]
	\label{a7}\: \neg P \= \Intro{\neg}\\
\end{proofbox}

b) $P \to (Q \to R), P, \neg R \vdash \neg Q$
\begin{proofbox}
	\label{b1}\: P \to (Q \to R) \= \textrm{premise} \\
	\label{b2}\: P \= \textrm{premise} \\
	\label{b3}\: \neg R \= \textrm{premise} \\
	\label{b4}\: Q \to R \= \Elim{\to}\ \ \ref{b1},\ref{b2}\\
	\label{b5}\: \neg Q \= \textrm{MT \ref{b3},\ref{b4}}\\
\end{proofbox}

\section{Problem 2}
a) $\neg P \to \neg Q \vdash Q \to P$\\
\begin{proofbox}
	\label{c1}\: \neg P \to \neg Q \= \textrm{premise}\\
	\[
		\label{c2}\: Q \= \textrm{assume}\\
		\label{c3}\: \neg \neg Q \= \Elim{\neg{\neg{}}}\ \ \ref{c2}\\
		\label{c4}\: \neg \neg P \= \textrm{MT \ref{c1},\ref{c3}}\\
		\label{c5}\: P \= \Elim{\neg{\neg{}}}\ \ \ref{c4}\\
	\]
	\label{c6}\: Q \to P \= \Intro{\to}
\end{proofbox}

b) $\neg P \lor \neg Q \vdash \neg (P \land Q)$\\
\begin{proofbox}
	\label{d1}\: \neg P \lor \neg Q \= \textrm{premise}\\
	\[
		\label{d2}\: \neg P \= \textrm{assume}\\
		\[
			\label{d3}\: P \land Q \= \textrm{assume}\\
			\label{d4}\: P \= \Elim{\land}\ \ \ref{d3}\\
			\label{d5}\: \bot \= \Elim{\neg}\ \ \ref{d2},\ref{d4}\\
		\]
		\label{d6}\: \neg (P \land Q) \= \Intro{\neg}\ \ \ref{d3}-\ref{d5}\\
	\]
	\[
		\label{d7}\: \neg Q \= \textrm{assume}\\
		\[
			\label{d8}\: P \land Q \= \textrm{assume}\\
			\label{d9}\: Q \= \Elim{\land}\ \ \ref{d8}\\
			\label{d10}\: \bot \= \Elim{\neg}\ \ \ref{d7},\ref{d9}\\
		\]
		\label{d11}\: \neg (P \land Q) \= \Intro{\neg}\ \ \ref{d8}-\ref{d10}\\
	\]
	\label{d12}\: \neg (P \land Q) \= \Elim{\lor}\ \ \ref{d1},\ref{d2}-\ref{d6},\ref{d7}-\ref{d11}\\
\end{proofbox}

c) $\neg P, P \lor Q \vdash Q$\\
\begin{proofbox}
	\label{e1}\: \neg P \= \textrm{premise}\\
	\label{e2}\: P \lor Q \= \textrm{premise}\\
	\[
		\label{e3}\: \neg Q \= \textrm{assume}\\
		\[
			\label{e4}\: P \= \textrm{assume}\\
			\label{e5}\: \bot \= \Elim{\neg}\ \ \ref{e1},\ref{e4}\\
		\]
		\[
			\label{e6}\: Q \= \textrm{assume}\\
			\label{e7}\: \bot \= \Elim{\neg}\ \ \ref{e3},\ref{e6}\\
		\]
		\label{e8}\: \bot \= \Elim{\neg}\ \ \ref{e3}-\ref{e7}\\
	\]
	\label{e9}\: Q \= \Intro{\neg}\\
\end{proofbox}

d) $P \lor Q, \neg Q \lor R \vdash P \lor R$
\begin{proofbox}
	\label{f1}\: P \lor Q \= \textrm{premise}\\
	\label{f2}\: \neg Q \lor R \= \textrm{premise}\\
	\[
		\label{f3}\: P \= \textrm{assume}\\
		\label{f4}\: P \lor R \= \Intro{\lor}_1\ \ \ref{f3}\\
	\]
	\[
		\label{f5}\: Q \= \textrm{assume}\\
		\[
			\label{f6}\: \neg Q \= \textrm{assume}\\
			\label{f7}\: \bot \= \Elim{\neg}\ \ \ref{f5},\ref{f6}\\
			\label{f8}\: P \lor R \= \Elim{\bot}\ \ \ref{f7}\\
		\]
		\[
			\label{f9}\: R \= \textrm{assume}\\
			\label{f10}\: P \lor R \= \Intro{\lor}_2\ \ \ref{f9}\\
		\]
		\label{f11}\: P \lor R \= \Elim{\lor}\ \ \ref{f6}-\ref{f8},\ref{f9}-\ref{f10}\\
	\]
	\label{f12}\: P \lor R \= \Elim{\lor}\ \ \ref{f1},\ref{f2},\ref{f3}-\ref{f4},\ref{f5}-\ref{f11}\\
\end{proofbox}

\section{Problem 3}
c is the number of binary connectives.
n is the number of negation connectives.
s is the number of propositional atoms or variables.
$\ell$ is the length of $\varphi$, not counting paraenthesis.\\
s = c + 1\\
n = $\ell$ - c - s = $\ell$ - c - (c + 1) = $\ell$ - 2c - 1
\section{Problem 4}

Base Case (1 term): \\
$\phi \equiv p$ \\
$\phi^* \equiv \neg p $\\
therefore: $\phi^* \equiv \neg \phi$\\

Defining $\phi^*$:\\
if $\phi \equiv \psi \lor \psi \equiv \phi^* \equiv \psi^* \land \psi^*$ \\ 
if $\phi \equiv \psi \land \psi \equiv \phi^* \equiv \psi^* \lor \psi^*$ \\ 
if $\phi \equiv \psi \equiv \phi^* \equiv \neg \psi^*$ \\ 

Induction Step: \\
Assume $\phi$ has more than 1 term and for all wff's $\psi$ with less terms than $\phi$ that $\psi^* = \neg \psi$\\
if $\phi$ is of the form $\psi \lor \psi$ then $\phi^* \equiv \psi^* \land \psi^* \equiv \neg \psi \land \neg \psi \equiv \neg (\psi \lor \psi) \equiv \neg \phi$ \\
if $\phi$ is of the form $\psi \land \psi$ then $\phi^* \equiv \psi^* \lor \psi^* \equiv \neg \psi \lor \neg \psi \equiv \neg (\psi \land \psi) \equiv \neg \phi$ \\
if $\phi$ is of the form $\neg \psi$ then $\phi^* \equiv \neg \psi^* \equiv \neg \neg \psi \equiv \neg \phi$ \\

Thus for any $\phi$, $\phi^* \equiv \neg \phi$.\\

\section{Problem 5}
\subsection{a)}
\{$\neg, \land$\}: $\phi \lor \phi \equiv \neg (\phi \land \phi)$, $\phi \to \phi \equiv \neg \phi \lor \phi \equiv \neg \neg \phi \land \neg \phi$ Anything with $\lor\  or \to$ can be represented by $\neg\  and\  \land$ as shown.

\{$\neg, \to$\}: $\neg \phi \to \phi \equiv \phi \lor \phi$, which means using $\neg$ and $\to$ you can represent $\lor$, and the set \{$\neg, \lor$\} is adequate as proved in the book.

\{$\neg, \bot$\}: $\neg \phi \equiv \phi \to \bot$ which means using $\neg$ and $\bot$ you can represent $\to$  and \{$\neg, \to$\} is adequate as proved above.
\subsection{b)}
If C doesn't contain $\neg \ or \bot$ then C $\subseteq$ \{$\land, \lor, \to$\}. There is no way to represent $\neg$ P for a formula with only P and the connectives in C, this C isn't adequate without $\neg \ or \bot$
\subsection{c)}
Truth Table for variables P and Q with the connectives \{$\neg, \leftrightarrow$\}\\
P \ \ Q \ \ $\neg$P \ \ $\neg$Q \ \ P$\leftrightarrow$Q \ \ $\neg$P$\leftrightarrow$Q \ \ P$\leftrightarrow \neg$Q \ \ $\neg$P$\leftrightarrow \neg$Q\\
T \ \ T \ \ \ F \ \ \ \ \ F \ \ \ \ \ T  \ \ \ \ \ \ \ \ \ F \ \ \ \ \ \ \ \ F \ \ \ \ \ \ \ \ \ \ \ \ \ T\\
T \ \ F \ \ \ F \ \ \ \ \ T \ \ \ \ \ F  \ \ \ \ \ \ \ \ \ T \ \ \ \ \ \ \ \ T \ \ \ \ \ \ \ \ \ \ \ \ \ F\\
F \ \ T \ \ \ T \ \ \ \ \ F \ \ \ \ \ F  \ \ \ \ \ \ \ \ \ T \ \ \ \ \ \ \ \ T \ \ \ \ \ \ \ \ \ \ \ \ \ F\\
F \ \ F \ \ \ T \ \ \ \ \ T \ \ \ \ \ T  \ \ \ \ \ \ \ \ \ F \ \ \ \ \ \ \ \ F \ \ \ \ \ \ \ \ \ \ \ \ \ T\\

In no case is it possible for an odd number of T's, which means P $\land$ Q can not be represented by only these 2 operators, so \{$\neg, \leftrightarrow$\} is not adequate.

\section{Problem 6}
Assume a formula with only 2 atoms so the possibilities are \{p, $\neg$p, q, $\neg$q\}.\\
The \# operator doesn't care about the order of the elements so the possibilities with only 2 atoms and their negations is \#($\phi, \phi, \psi$) or \#($\phi, \neg \phi, \psi$).\\
 \#($\phi, \phi, \psi$) $\equiv \phi$\\
 \#($\phi, \neg \phi, \psi$) $\equiv \psi$\\
 $\phi$ and $\psi$ are both one of the possible atoms, and their negations are also in that set. So the outcome is always equivalent to some single term.
Thus, for example, it's impossible to represent p $\land$ q with only \{$\neg$, \#\}\\.

\end{document}  