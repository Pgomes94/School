\documentclass{article}
\usepackage{amsmath,amsthm,amsfonts, amssymb}
\usepackage{hyperref}
\usepackage{tikz}
\usetikzlibrary{arrows,automata}
\usetikzlibrary{decorations.text,calc}

\setlength{\textheight}{8.5in}
\setlength{\evensidemargin}{0.0in}
\setlength{\oddsidemargin}{0.0in}
\setlength{\topmargin}{-0.5in}
\setlength{\textwidth}{6.5in}

\usepackage{prooftree}
\usepackage{boxproof}
\usepackage{parskip}

\newcommand{\handout}[6]{
   \renewcommand{\thepage}{#1-\arabic{page}}
   \noindent
   \begin{center}
      \vbox{
    \hbox to \textwidth { #2 \hfill #3 }
       \vspace{4mm}
       \hbox to \textwidth { {\Large \hfill #4  \hfill} }
       \vspace{4mm}
       \hbox to \textwidth { { #5 \hfill #6} }
      }
   \hrulefill
   \end{center}
   \vspace*{4mm}
}
\begin{document}
\handout{}{CS 512: Formal Methods}{Spring 2016}{Assignment 8: LTL, CTL, Traces and Execution Paths in Transition Systems}{Instructor: Assaf Kfoury}{Author: Patrick Gomes}
\section{Problem 1}
a) \\
No. Starting from $s_0 , \pi \vDash \{s_0, (s_3, s_4, s_0)^* \}$ \\
\\
b) \\
Yes. There is no infinite path that does not pass a state where $\neg p$ is true at one point.\\
\\
c) \\
This is also true. Starting from $s_0$ you can either go to $s_1$, which satisfies the condition (P $\land$ X P), or you go to $s_3,$ and then $s_4, s_0$ which also satisfies the condition. So there is no path that doesn't satisfy the condition at least once in the future.\\

\section{Problem 2}
a) \\
No for the same reason as 1.a, you can go down the same path $\pi \vDash \{s_0, (s_3, s_4, s_0)^* \}$ \\
\\
b)\\
\\
Yes for the same reason as 1.b, there is no infinite path that does not pass a state where $\neg p$ is true at one point.\\
\\
c)\\
No, because the $\forall$Xp will not be satisfied if you go from $s_0$ to $s_3$ because the immediate next is not also true.\\

\section{Problem 3}
As seen in problems 1.c and 2.c, directly substituting a $\forall$ will not translate LTL to CTL. The LTL formula F X p is satisfiable because there will be some future state where p is true no matter what path is taken. On the other hand, the CTL formula $\forall$ F X p is not satisfiable because from the initial state you can go to $s_3$ which is $\neg$ p and violates the forall condition.\\

\section{Problem 4}
a)\\
Infinite loop back to start : (abb)\\
Reaching end states : (ab)\\
You can also reach end state $s_2$ by repeating a*\\
Combined all together = (abb)*aba*\\
\\
b)\\
Infinite loop back to start : (abb)\\
reaching end states : (ab)\\
continuing at state $s_2$ : a\\
continuing at state $s_4$ : (bab)\\
Combined all together = (abb)*ab(a + bab$)^w$\\
\\
c)\\
Yes there is a path there. At some future point, p implies the next next state will be p forever. An infinite path $\pi \vDash ab(a)^w$ satisfies this condition.\\ 
The trace for that path is pp($\neg p)^w$\\

\end{document}