\documentclass{article}
\usepackage{amsmath,amsthm,amsfonts, amssymb}
\usepackage{hyperref}
\usepackage{tikz}
\usetikzlibrary{decorations.text,calc}


\setlength{\textheight}{8.5in}
\setlength{\evensidemargin}{0.0in}
\setlength{\oddsidemargin}{0.0in}
\setlength{\topmargin}{-0.5in}
\setlength{\textwidth}{6.5in}

\usepackage{prooftree}
\usepackage{boxproof}
\usepackage{parskip}

\newcommand{\handout}[6]{
   \renewcommand{\thepage}{#1-\arabic{page}}
   \noindent
   \begin{center}
      \vbox{
    \hbox to \textwidth { #2 \hfill #3 }
       \vspace{4mm}
       \hbox to \textwidth { {\Large \hfill #4  \hfill} }
       \vspace{4mm}
       \hbox to \textwidth { { #5 \hfill #6} }
      }
   \hrulefill
   \end{center}
   \vspace*{4mm}
}
\begin{document}
\handout{}{CS 512: Formal Methods}{Spring 2016}{Assignment 6: Linear Temporal Logic}{Instructor: Assaf Kfoury}{Author: Patrick Gomes}
\section{Problem 1}
a) \\
\begin{align*}
	\varphi_1(x) \triangleq (1 < x)\ \land \forall y (\exists z (y \times z = x) \to (y=1 \lor y=x))\\
	\varphi_2(m,n) \triangleq (1 < m) \land (1 < n) \land \forall y (\exists z(y \times z=m) \to (\neg \exists v (y \times v = n)\\
	\varphi_3(X) \triangleq \forall x (X(x) \to \exists y(y \times (1+1) = x)\\
	\varphi_4(X) \triangleq \forall x (X(x) \to (x=1 \lor \exists y (X(y) \land (x = y * (1+1))))
\end{align*}

\section{Problem 2}
\begin{align*}
	\text {Formula: $\neg$p U (F r $\lor$ G $\neg$q $\to$ q W $\neg$r)}\\
	\text{Subformulas:}\\
	p \\
	\neg p \\
	r\\
	F\ r\\
	q\\
	\neg q\\
	\text{G $\neg$ q}\\
	\text{F r $\lor$ G $\neg$ q}\\
	\neg r\\
	\text{q W $\neg$ r}\\
	\text{F r $\lor$ G $\neg$ q $\to$ q W $\neg$ r}\\
	\text{$\neg$ p U F r $\lor$ G $\neg$ q $\to$ q W $\neg$ r}	\\
\end{align*}

\section{Problem 3}
a) G\ a\\
$\pi$ = $(q_3,q_4)^i$\\
$\mathcal{M}, q_3 \nvDash \phi$ \text{A counter example: $\pi$ = $q_3,q_2$. $q_2$ is $\neg a$}

b) a\ U\ b\\
$\pi$ = $q_3,q_2$\\
$\mathcal{M}, q_3 \nvDash \phi$ \text{A counter example: $\pi$ = $q_3,q_1$. $q_1$ is $\neg a$ but b is not true.}

c) a\ U\ X($a \land \neg b$)\\
$\forall i \geq	 1\ \pi^i$ = $q_3,q_4,q_3$\\
$\mathcal{M}, q_3 \nvDash \phi$ \text{A counter example: $\pi$ = $q_3,q_2$. $q_2$ is $\neg a$ and there is no way to get to $q_3$ and satisfy the next clause. }

d) X $\neg b\ \land$\ G($\neg a \lor \neg b$)\\
$\pi$ = $q_3,q_1,q_2^i$\\
$\mathcal{M}, q_3 \nvDash \phi$ \text{A counter example: $\pi$ = $q_3,q_4$. $q_4$ does not satisfy the next clause}\\

e) X (a $\land$ b) $\land$ F($\neg a \land \neg b$) \\
$\pi$ = $q_3,q_4,q_3,q_1$\\
$\mathcal{M}, q_3 \nvDash \phi$ \text{A counter example: $\pi$ = $q_3,q_2$. $q_2$ does not satisfy the next clause.}

\section{Problem 4}
\begin{align*}
	\pi \vDash (X \varphi) U (X \psi) \ \ \text{iff (By expanding definition of U)}\\
	\pi \vDash (\exists i \geq 1) (\pi^i \vDash X\psi \land (\forall j < i) (\pi^j \vDash X\varphi)) \ \ \text{iff (By removal of X and representing next state, not current.)}\\
	\pi^2 \vDash (\exists i \geq 1) (\pi^{i+1} \vDash \psi \land (\forall j < i) (\pi^{j+1} \vDash \varphi)) \ \ \text{iff (By condensing definiton of U)}\\
	\pi^2 \vDash \varphi U \psi \ \ \text{iff (By reintroducing X to represent current state.)}\\
	\pi \vDash X (\varphi U \psi)\\
\end{align*}

\end{document}