\documentclass{article}
\usepackage{amsmath,amsthm,amsfonts}
\usepackage{hyperref}
\usepackage{tikz}
\usetikzlibrary{decorations.text,calc}


\setlength{\textheight}{8.5in}
\setlength{\evensidemargin}{0.0in}
\setlength{\oddsidemargin}{0.0in}
\setlength{\topmargin}{-0.5in}
\setlength{\textwidth}{6.5in}


\newcommand{\handout}[6]{
   \renewcommand{\thepage}{#1-\arabic{page}}
   \noindent
   \begin{center}
      \vbox{
    \hbox to \textwidth { #2 \hfill #3 }
       \vspace{4mm}
       \hbox to \textwidth { {\Large \hfill #4  \hfill} }
       \vspace{4mm}
       \hbox to \textwidth { { #5 \hfill #6} }
      }
   \hrulefill
   \end{center}
   \vspace*{4mm}
}

\begin{document}

\handout{}{CS 512: Formal Methods}{Spring 2016}{Assignment 2: Background to SAT/SMT Solvers}{Instructor: Assaf Kfoury}{Author: Patrick Gomes}

\section{Problem 1}
$| | \ : \ \mathcal{T} \to \mathbb{N} $
\begin{align*}
| t | = 
\begin{cases}
	1 + | Lt | + | Rt | \ \ \ \ if \ \ t = < Lt, Rt > \\
	0\ \ \ \ \ \  otherwise\\
\end{cases}
\end{align*}
$height \ : \ \mathcal{T} \to \mathbb{N} $
\begin{align*}
heigh(t) = 
\begin{cases}
	max(height(Lt), height(Rt)) + 1 \ \ \ \ if \ \ t = < Lt, Rt >\\
	0 \ \ \ \ \ \ otherwise\\
\end{cases}
\end{align*}

\section{Problem 2}
(a)\\ \\
Poset:
\begin{align*}
	\text{Division within the natural numbers has the Reflexive, Anti-symmetric and Transitive properties.}
\end{align*}
Least Upper Bound:
\begin{align*}
	\text{Least Common Multiple of a,b $\in \mathbb{N}$ -\{0\}	 returns the smallest number, c, such that both  a $\unlhd$ c}\\  
	\text{and b $\unlhd$ c. This is the smallest multiple that both numbers divide, so $\lor$ (lcm) is the Least Upper Bound.}
\end{align*}
Greatest Lower Bound:
\begin{align*}
	\text{Greatest Common Divisor of a,b $\in \mathbb{N}$ - \{0\} returns the largest number, c, such that both c $\unlhd$ a and c $\unlhd$ b}\\
	\text{This is the largest number that divides both numbers, so $\land$ (gcd) is the Greatest Lower Bound.}
\end{align*}

Thus $\mathcal{A}$ satisfies all the properties of a lattice.\\ 

(b)
\text{$\mathcal{A}$ is a distributive latice. Proof:} \\
\text{Any number can be represented as the product of primes (prime factorization theorem)}\\  
\text{Represent a, b and c as the product of all primes (1 $\dots$ n) raised to a power.}\\
\text{Example: a = ($p_1^\alpha, \dots, p_n^\alpha$) where each $\alpha$ represents the number of times that prime appears in a's prime factorization}\\
\text{Repeat the same for b and c.}\\
\text{Represent lcm ($\lor$) as the max function and gcd ($\land$) as the min function}\\
\text{Proving a $\land$ (b $\lor$ c) = (a $\lor$ b) $\land$ (a $\lor$ c) :}\\
\text{a $\land$ (b $\lor$ c) = $p_1^{min\{\alpha, max\{\beta, \gamma\}\}} \dots p_n^{min\{\alpha, max\{\beta, \gamma\}\}}$ }\\
\text{(a $\lor$ b) $\land$ (a $\lor$ c) = $p_1^{max\{min\{\alpha, \beta\}, min\{\alpha, \gamma\}\}} \dots p_n^{max\{min\{\alpha, \beta\}, min\{\alpha, \gamma\}\}} $}\\
\text{Any ordering of $\alpha, \beta, \gamma$ will result in the same output for both equations (not shown to save space).}\\
\text{The same strategy and proof will satisfy the other constraint of: a $\lor$ (b $\land$ c) = (a $\lor$ b) $\land$ (a $\lor$ c)}\\

\section{Problem 3}
Proof by induction: 
\text{Base Case (1 Connective):}\\
\text{$\neg p = \neg p$}\\
\text{$\neg (p \lor q) = \neg p \land q$} \\
\text{$\neg (p \land q) = \neg p \lor q$} \\ \\
\text{Induction Step:}\\
\text{A WFF $\phi$ with n+1 connectives can be represented in one of the following forms: }\\
\begin{equation*}
	\begin{cases}
		\text{$\neg \phi$}\\
		\text{$\phi \lor \phi$}\\
		\text{$\phi \land \phi$}\\
	\end{cases}\\ 
\end{equation*}
\text{The last 2 cases are valid in NNF. The first case is proved through De Morgan's.}\\
$\neg (\phi \land \phi) \equiv \neg \phi \lor \neg \phi$ \text{The left hand side has 4 terms. The right hand side has 5. 5 $<$ 1.5 * 4}\\
\text {The same situation happens with $\lor$ instead of $\land$}. 


\section{Problem 4}
\text{Extending the HORN Clause will cause the HORN algorithm to say that certain formulas} \\
\text{are satisfiable even if they aren't.}\\
\text{Here is one such example:}\\
$(\top \to p) \land (\top \to \neg p)$\\
\text{This is a valid HORN Formula consisting of two HORN clauses.}\\
\text{The algorithm will assign true to both p and $\neg$ p, which can't be true.}\\
\text{Therefore, adding $\neg$ to HORN Clauses will break the algorithm.}\\

\end{document}